\section{Code Framework}


%%%%%%%%%%%%%%%%%%%%%%%%%%%%%%%%%%%%%%%%%%%%%%%%%%%%%%%%%%%%%%%%%%%%%%%%
% Classes
%%%%%%%%%%%%%%%%%%%%%%%%%%%%%%%%%%%%%%%%%%%%%%%%%%%%%%%%%%%%%%%%%%%%%%%%

\subsection{Classes}
There are a bunch of existing classes which should be used by everyone to setup
the interfaces among the groups.
The following table gives an overview and short description of each class.

\noindent
\begin{tabular}{|l|l|}
	\hline
	CDataArray2D.hpp				&
		Data storage for 2D arrays							\\
	\hline
	CDataDrawingInformation.hpp		&
		Data storage for interactive drawing information	\\
	\hline
	CGlTexture.hpp					&
		Abstraction for OpenGL Textures						\\
	\hline
	CParameters.hpp					&
		Program and simulation parameters					\\
	\hline
	CPipelinePacket.hpp				&
		Pipeline packet capable of being forwarded via the pipeline	\\
	\hline
	CPipelineStage.hpp				&
		Pipeline stage providing interfaces for pipelining	\\
	\hline
	CSDLInterface.hpp				&
		SDL Interface for visualization and interactivity	\\
	\hline
	CStage\_ImageInput.hpp			&
		Pipeline stage for image input (single image from file)	\\
	\hline
	CStage\_ImageProcessing.hpp		&
		Pipeline stage for image processing filter			\\
	\hline
	CStage\_VideoInput.hpp			&
		Pipeline stage for video input, e.g.\,from webcam	\\
	\hline
	CStage\_VideoOutput.hpp			&
		Pipeline stage for video output				\\			
	\hline
	main.cpp						&
		main entry to setup pipelined scenarios		\\
	\hline
\end{tabular}


%%%%%%%%%%%%%%%%%%%%%%%%%%%%%%%%%%%%%%%%%%%%%%%%%%%%%%%%%%%%%%%%%%%%%%%%
% Pipeline
%%%%%%%%%%%%%%%%%%%%%%%%%%%%%%%%%%%%%%%%%%%%%%%%%%%%%%%%%%%%%%%%%%%%%%%%
\subsection{Pipeline}
\label{sec:pipeline}

The idea of a pipelining model is creating independent execution parts with
particular input-output specifications.
E.g.\,a webcam only provides images which are forwarded to the image filter.
After processing of the image filter, this information is further forwarded to
the simulation (not yet implemented) and then to the output for visualization.

All those pipeline stages are independent and thus can be independently
processed - e.g. Image filter and simulation computations in parallel.

%%%%%%%%%%%%%%%%%%%%%%%%%%%%%%%%%%%%%%%%%%%%%%%%%%%%%%%%%%%%%%%%%%%%%%%%
% Pipeline stage
%
\subsubsection{Pipeline stage}

We continue with an example given by the image processing filter
\textit{CStage\_ImageProcessing.hpp}.
For well-known interfaces, each new pipeline stage has to inherit the class
CPipelineStage:

\begin{lstlisting}
class CStage_ImageProcessing	:	public
	CPipelineStage
...
\end{lstlisting}

\noindent
For processing of the images, paramters are required to know which
computations to do, at least a single input storage is required as well as an
output storage to forward processes images to other classes:

\begin{lstlisting}
...
/**
 * global parameters
 */
CParameters &cParameters;

/**
 * input image
 */
CDataArray2D<unsigned char,3> input_cDataArray2D;

/**
 * processed image
 */
CDataArray2D<unsigned char,3> output_cDataArray2D;
...
\end{lstlisting}

\noindent
Since the parameters are shared with the other classes, they are
setup in the constructor:

\begin{lstlisting}
public:
/**
 * constructor
 */
CStage_ImageProcessing(CParameters &i_cParameters):
CPipelineStage("ImageProcessing"),
cParameters(i_cParameters)
{
}
\end{lstlisting}

\noindent
In case of an input sent via the pipeline of another stage such as the video
input, the method \textit{pipeline\_process\_input} is executed and has to be
implemented. This interface is particularly requested by the class
\textit{CPipelineStage}.

\begin{lstlisting}
void pipeline_process_input(
	CPipelinePacket &i_cPipelinePacket
)
{
...
\end{lstlisting}

\noindent
Since not all possible data types can be probably processed, we have to check
for compatible input packages and unpack the data to make it available with our
accessor class:
\begin{lstlisting}
// we are currently only able to process "unsigned char,3" data arrays.
if (i_cPipelinePacket.type_info_name != typeid(CDataArray2D<unsigned char,3>).name())
{
	std::cerr << "ERROR: Video Output is only able to process (char,3) arrays" << std::endl;
	exit(-1);
}

// unpack data
CDataArray2D<unsigned char,3> *input = i_cPipelinePacket.getPayload<CDataArray2D<unsigned char,3> >();
\end{lstlisting}

\noindent
After unpacking the data and processing the data with more details available in
the source code file itself, the output data array is pushed to the pipeline and
thus forwarded to the next pipeline stages:
\begin{lstlisting}
CPipelineStage::pipeline_push((CPipelinePacket&)output_cDataArray2D);
\end{lstlisting}

%%%%%%%%%%%%%%%%%%%%%%%%%%%%%%%%%%%%%%%%%%%%%%%%%%%%%%%%%%%%%%%%%%%%%%%%
% Pipeline setup
%
\subsubsection{Pipeline setup}
After programming several stages, their input and output has to be
connected after instantiation.
E.g.\,let us assume that we like to have an static input image with an image
filter and the possibility to draw into the image, this would lead to the
following pipeline:

\begin{lstlisting}
// static image input
CStage_ImageInput cStage_ImageInput(cParameters);
// video output
CStage_VideoOutput cStage_VideoOutput(cParameters);

// PIPELINE CONNECTIONS
// forward image to video output
cStage_ImageInput.connectOutput(cStage_VideoOutput);
// forward mouse movements to image input
cStage_VideoOutput.connectOutput(cStage_ImageInput);


// initial push of static image
cStage_ImageInput.pipeline_push();

// main loop
while (!cParameters.exit)
{
	// trigger image input to do something
	cStage_VideoOutput.main_loop_callback();
}
\end{lstlisting}

\noindent
For our pipeline concept, only the outputs have to be connected.
The initial push for the image input is required to initially forward the static
image to the video output.

The main loop is required to e.g.\,check for user input, to draw updates for the
visualization and to run a simulation timestep.


\subsection{Interaction}
Several keystrokes currently exist updating some parameters in the class
\textit{CParameters}.
Note that all pipeline stages get a reference to this class during
initialization.

Since the Video output is closely connected to the input system, all
input keystrokes which are not directly processed are forwarded to the method
key\_down of the parameter class:

\begin{lstlisting}
/**
 * return bool if processed
 */
bool key_down(char i_key)
{
	switch(i_key)
	{
	case SDLK_j:
		stage_imageprocessing_filter_id--;
		std::cout << "Using filter id " << stage_imageprocessing_filter_id << std::endl;
		return true;

	case SDLK_k:
		stage_imageprocessing_filter_id++;
		std::cout << "Using filter id " << stage_imageprocessing_filter_id << std::endl;
		return true;
\end{lstlisting}

\noindent
This allows the modification of the parameters during the programs runtime.
So far the following keystrokes are defined:

\noindent
\begin{tabular}{|c|l|}
	\hline
	\multicolumn{2}{|l|}{\textbf{General}}	\\
	\hline
	q & quit program	\\
	\hline
	\hline

	\multicolumn{2}{|l|}{\textbf{1: image Processing}}	\\
	\hline
	j,k & decrease / increase filter id\\
	\hline
	g,t & decrease/increase threshold value\\
	\hline
\end{tabular}


\subsection{Program start}

Several program parameters currently exist and are also processed in
CParameters:

\noindent
\begin{tabular}{|c|l|}
	\hline
	\multicolumn{2}{|l|}{\textbf{General}}	\\
	\hline
	p	& pipeline id to use (see main.cpp)	\\
	\hline
	v	& verbosity level	\\
	\hline
	\hline

	\multicolumn{2}{|l|}{\textbf{0: image/videoinput}}	\\
	\hline
	d	& video device string to use\\
	\hline
	w	& request this width for video input\\
	\hline
	h	& request this height for video input\\
	\hline
	i	& path to input image to use\\
	\hline
	\hline

	\multicolumn{2}{|l|}{\textbf{3: fluid simulation LBM}}	\\
	\hline
	v	& switch between flag field and velocity output	\\
	\hline
	\hline
	
	\multicolumn{2}{|l|}{\textbf{4: parallelization}}	\\
	\hline
	n	& number of threads to use\\
	\hline
	\hline

\end{tabular}
\newpage