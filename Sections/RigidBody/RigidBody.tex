\section{Rigid Body Simulation}
\tododone[inline]{Benni}{responsible for section \textbf{RB}: Friedrich}

COLLIDE! and resolve correctly... but mainly COLLIDE!

\begin{itemize}
\item simulation driven game one half LBM, one half RB
\item RB simulation is very simple for circles, becomes more complicated for rectangles and gets a real problem for polygons and other complex stuff. Our strategy: Wrap a fully developed RB engine and use its functionality
\item the main problem is not the implementation of the functionality, but the interfaces!
\end{itemize}

\subsection{Bullet}
\subsubsection{Building}
To build bullet, you need:
\begin{itemize}
\item cmake
\end{itemize}
Instructions:
\begin{enumerate}
\item Extract the \verb+fa/sources-for-rigidbody/bullet3-2.83.6.tar.gz + somewhere.
\item run \verb+$cmake . + in the top directory.
\item run \verb+$make -j4+ (\verb+-j4+ chooses 4 threads for compilation) in the top directory.
\item run \verb+$sudo make install + in the \verb+src+ directory.
\item Enjoy! (\& profit)
\end{enumerate}
\subsubsection{Functionality we used}
\begin{itemize}
\item Only explain bullet's functionality, we actually used
\item also explain our pitfalls (there have been plenty situations where we thought: come on bullet... that just doesn't make sense...)
\end{itemize}

\subsubsection{Discussion}
\tododone[inline]{Benni}{responsible for \textbf{Discussion} on RB and Bullet: Erik, Friedrich, Benni}
\begin{itemize}
\item was a bit of an overkill: in the end we only used circles anyhow.
\item better way: 
\begin{itemize}
\item first simple mock up with basic, easy functionality implemented on our own: explicit euler, circles, no performance optimization...
\item then wrap physics engine as soon as interfaces are defined and running, whole change under the hood
\end{itemize}
\end{itemize}

\subsection{Interfaces}
\tododone[inline]{Benni}{responsible for subsection \textbf{Interfaces} of RB: Friedrich}
\begin{itemize}
\item RB have many interfaces! Most complex to LBM, but also many more interfaces with different demands.
\item discrete version(LBM, ImgRecognition) vs. continuous version(Visu, RB Internal) of RBs
\end{itemize}
\subsubsection{to Input Devices}
\tododone[inline]{Benni}{responsible for interface from RB \textbf{to Input Devices}: Erik?}
\begin{itemize}
\item how to model rowing
\item apply forces to bodies depending on input
\item in the end more time for finetuning would have been good!
\end{itemize}
\subsubsection{to Image Recognition}
\tododone[inline]{Benni}{responsible for interface from RB \textbf{to Image Recognition}: Friedrich}
\begin{itemize}
\item how to model boundaries
\item how to create different rigid bodies
\end{itemize}
\subsubsection{to Visualization}
\tododone[inline]{Benni}{responsible for interface from RB \textbf{to Visualization}: Benni?}
\begin{itemize}
\item represent shapes in an abstract way
\end{itemize}
\subsubsection{to LBM}
\tododone[inline]{Benni}{responsible for interface from RB \textbf{to LBM}: Erik?}
\begin{itemize}
\item how to send important quantities (discretization of continuous RB into cells, velocity of RB...)
\item how to receive important quantities (forces and torques from the fluid on the body)
\end{itemize}