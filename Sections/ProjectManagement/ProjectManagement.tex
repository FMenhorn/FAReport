\section{Project Management}
\tododone[inline]{Mohamed}{\textbf{Project Management}}
\todointern[inline]{Benni}{Someone proofreading?}
This section will give an insight about the recommended management strategy to be used for this kind of projects and the most probable challenges that might show up along the way to be avoided. It will provide an overview of the necessary managerial tools suggested for similar projects in the future.


\subsection{Agile Management}
%\textit{Here a brief introduction on agile management will be given with minimal theoretical details (since we always minimze in agile)}
Agile management is a sort of management commonly-used in software development projects, since it is more flexible to developers changing decisions and customers updated needs. The principle of such a management scheme is to divide the full project into smaller chunks, referred to as sprints. Each chunk involves finishing a functioning part of the project. For example, in our project theme, some possible sprints could have been:
\begin{enumerate}
  \item{Setting up all interfaces with dummy data to close the pipeline}
  \item{Replace the dummy data with real data via implementing the necessary models or wrapping the available libararies}
  \item{Adding different levels to the game via image processing + implementing a more complex RB representation}
  \item{Allowing the user to draw his field}
  \item{etc...}
\end{enumerate}
It can be observed, that for every sprint, the entire code is running, integrated and producing some output (despite its realisticity). This output keeps developing and approaching its final shape for every sprint. In addition, the developers don't fall in the trap of being involved in their part while thinking that at some distant point in time, they will merge it with the other teams' and everything will run smoothly. Moreover, this method allows for continuous unit-testing(explained at \ref{unit_testing}) and early errors correction.
One favorable aspect about agile management also is the flexibility in human resources management through relocation and colocation. Relocation is the shuffling of one person's role from one team to another after completing his task there or if his team is no more in need of all the current manpower. Colocation, on the other hand, is the presence of one member among more than one team of developers, which is specially encouraged for people handling data interfaces between different teams.


\subsection{Useful Organizational Tools}
In this section, the light will be shed on the methodology adopted by our group for deciding the project's theme. Also the concept of scrumming will be exposed. In addition, other facilitation tools will be presented.

\subsubsection{Project's Selection For Dummies}
Project selection is the first step in implementing it. With the flow of ideas and a wide spectrum of imagination among the developers on their first gatherings, they are suddenly confronted with an as-wide range of projects ideas, that for the first glance seem appealing. however, project selection is more than just the vibrance of the idea. The following section will detail some tips on guiding the project selection.

\begin{itemize}
  \item \textbf{Brainstorming}: The first step in choosing a project is to set the sky as the limit for people to pour out their ideas all together in one pool, and most importantly, out-loud, allowing people to generate more ideas and develop one another's. An important aspect to be aware of is not to let this stage take a very long time, since time in Ferienakademie is somehow restricted. Also, it would be very helpful to try to find common grounds between different ideas as soon as possible. For example, "all the proposed ideas will require a Kinect input interface" or "regardless the topic we eventually select, we will need an RB-LBM coupling framework". Reaching such conclusions early initiates the project in concurrancy with finalizing/elaborating the ideas.
  
  \item \textbf{Selection Matrix}: It is very crucial and helpful to decide upon some criteria to evaluate the "goodness" of the idea. It will be found that all ideas are creative and extravagent and one would wish to have such an output, but their charm might dim after being judged subjectively against criteria such as:
  \begin{itemize}
    \item \textit{Usability}: How easy the end-user can play this game or use this program? Will he/she need an intensive documentation or a direct supervised training or instructions to start using it correctly ?
    \item \textit{Extensability}: How far can we go beyond the first milestone ? Can we add more features to the project to make it more challenging ?
    \item \textit{Audience-interest}: How attractive it is for audience to watch the game ? For example, chess is a big fail even if you like it !!
    \item \textit{Developers-interest}: Are we eager to put all our effort for around 10-days on that project ?
  \end{itemize}
  For our project, having such a matrix (with projects' proposals on one edge and criteria on the other), the developers simply voted on each criteria for every project and points we counted. Automatically, some projects dropped out and it was obvious towards which idea(s) is the decision heading. In case more than one project got close high ranks, their ideas might be rediscussed and re-evaluated again for the final decision.
\end{itemize}


\subsubsection{Scrumming}
%\textit{An idea about scrumming, what are they used for, the sort of updates discussed through them, etc.}
The term scrumming in agile management is inspired by the first move in an American football game; where all the players from both teams initially start at the centerline then in a glance spread all over the field - that move is the \emph{scrum}. The idealisation of such a concept in software development is such that the developers meet frequently (weekly or biweekly) to discuss the following three points:
\begin{itemize}
  \item{What has been achieved/completed since the last scrum ?}
  \item{What is the prospective task until the next scrum ?}
  \item{Has any challenges been faced that require help from other developers ?}
\end{itemize}
The main idea of scrumming is to keep all developers and the mangaer(s) aware of the current status of each separate area in the project and what has each team achieved so far. Scrums are neither for feedback on the project or the developers nor for brainstorming. In our case, the scrum was held almost every morning, and if needed, it was rescheduled to a different time during the day, but generally, a daily scrum.
Our team consisted of about 20 developers and having all those in one meeting was thought to be hectic. Hence, for every team, a representer was chosen to attend the scrum and update on behlaf of his team. Nevertheless, involving the entire team in the meeitng doesn't sound like a bad idea since it could provide an even more detailed insight about the progress of each team (versus the general statement about the progress given by the representer), and enhances a more-direct communication channel between members from different teams involved on common issues (such as interfaces). In our project, I guess, it would have helped to have the LBM developers listen directly in scrums from the RB people and vice versa.


\subsubsection{Code Handling Facilitators}
One of the duties of the project management is to ensure that code is being handled smoothly within the team. Our team has found the following tools pretty beneficial while working on the project.

\begin{itemize}
  \item \textbf{Git}\footnote{An alternative code management tool could be SVN}: for organized code sharing without conflicts, repository updating with current work, experimenting safely with developers' ideas on separate branches, managing conflicts between coders' work, etc.
  \item \textbf{CMake}: for a more neat and modular compilation of the whole project from one parent directory
  \item \textbf{Valgrind}: for debugging the code and hunting malicious code lines that result in segmentation faults and other nasty bugs
  \item \textbf{Unit-testing}\footnote{Boost provides a good unit testing tool for C++}: testing each block of code or every interface separately with hardcoded example data to make sure it is bug free and interacting safely with given input and providing sensible output before handling it to the other teams up- and downstream the pipeline\label{unit_testing}
  \item \textbf{Code References}: since we were in an internet-deserted area, having an offline chest of language references and library documentation is worth a fortune. Useful documentations are library's supplier-provided documentation, c++ reference and an offline repository of stackoverflow (and might as well for other libraries used if no PDF documentation was found)
\end{itemize}

\subsection{Challenges and Obstacles}
\todourgent[inline]{Benni}{Move this part in a final summary chapter? Summary and lessons learned}
\tododone[inline]{Benni}{\textbf{Challenges and Obstacles}: did proofreading}
Challenges are not bugs, they are features, and one should be indeed aware of their existence and do his/her best to avoid them or at worst be ready to handle them if they can't be detoured \todointern{Benni}{do your best to avoid challenges! Surprising attitude for a BGCE student :P}. Some challenges that we experienced through our project are listed below.

\begin{itemize}
  \item \textbf{Time constraint}: We had many hikes, plus there were a couple of social gatherings and events such as the tournaments and the cultural evening. This boiled down two weeks of work dedicated to the project to much less time. This aspect needs to be taken into account when setting the milestones for the project.
  \item \textbf{Defining the project's topic}: Due to the multi-disciplinary background of the participants as well as various interests, the brainstorming session turns out really vibrant between different ideas. Attention should be drawn to common grounds and the manager(s) should at some point occlude the ongoing flow of ideas to agree onto something.
  \item \textbf{Defining the milestones (be really to \todo{?} agile)}: One of the challenges faced in our projects was phrasing the first milestone deliverables thorough-enough. The first milestone deliverable should not have been just about "Create a game with the simplest fluid-structure interaction and only taking input from Kinect". Such a statement comprised too much of details to work on such as the level of details of rigid bodies capabilities. In addition, it required a comprehensive data handling system between different pipeline stages, which implied that each team should necessarily implement fully-working functionality in order to test the pipeline integrity (imagine how long could this have taken !!).
  A more efficient/agile approach would have been to set the first milestone deliverable to be "building and closing the pipeline with dummy data". This would have crossed out a milestone from the list and given all the teams an insight about into the interaction with the preceding and proceeding pipeline stages.
  \item \textbf{Testing the code}: With around 20 developers, each one implementing a chunk, the code is very vulnerable to get broken by some mathematical fault in a function or a segmentation-fault in one group's code. This could be avoided -- up to some extend -- by unit-testing. Unit-testing is hard-coding an example in order to test the implemented methods and ensure that they are giving the expected response. Although unit-testing requires a considerable additional workload, it, on the other hand, saves the programmers a lot of debugging time in case something goes wrong. This was extremely crucial in the interface between rigid bodies and LBM due to the massive amount of communication between the two codes.
  \item \textbf{Waterfalling}: Waterfall differs from agile management in that it treats the project as a one whole and the tangible output is only delivered at the end. This could be inefficient in projects like ours since waiting till the end since the developers might get demotivated as time passes without tangible output\todo{?}. Also, the project becomes more rigid to adding or modifying milestone deliverables for later stages. The risk of waterfalling is usually high when the first milestone is not clearly bounded and be made aware to the developers. The risk evolves that they end up indulging into coding and extending functionality of their blocks, while delaying integration of their respective parts with the others on the pipeline.
  \item \textbf{Speaking German}: Although the course's official language is English, nevertheless, one can't hold a group of 90\% of Germans to speak their mother language. In such an environment, non-formal meetings between developers (meaning small talks during coffee breaks, discussion over lunch, casual chats, etc.) encourage ideas flow and facilitate problem solving among developers as well as helps management to be updated with deeper details of the project's status. It is rather recommended that the managers' language skills in later courses is good enough to engage and comprehend such situations easily and not just rely on the English scrum meeting for udpates.
\end{itemize}

\todoinintern{}{additional stuff for lessons learned}{ 
\begin{itemize}
\item Managers should not distribute among groups, but manage the team and mistrust everybody. 
\item A group dedicated to game logic (became part of rigid bodies job...) and level design would have been good. 
\item Visu people used OpenGL (very rigid, only runs on some computers, BUT powerful), SDL was the intended language (default in sceleton). Maybe presentation on SDL would have been good? Are there also other possible presentation topics, we missed during ferienakademie?
\end{itemize}
}

