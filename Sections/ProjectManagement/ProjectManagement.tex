\section{Project Management}
% \tododone[inline]{Mohamed}{\textbf{Project Management}}
% \tododone[inline]{Benni}{did proofreading}
This section will give an insight into the recommended management strategy to be used for this kind of project and give an overview over the necessary managerial tools suggested for similar projects in the future.
In addition, a follow-up section at the end of this document will summarize the the management-oriented mistakes fallen into and challenges faced by different teams.

\subsection{Agile Management}
%\textit{Here a brief introduction on agile management will be given with minimal theoretical details (since we always minimze in agile)}
Agile management is a management strategy which is commonly used in software development projects. It is very flexible with respect to developers changing decisions and customers updating their needs. The principle of this management scheme is to divide the full project into smaller chunks, referred to as sprints. Each chunk involves finishing a fully working part of the project. For example, in our project theme, a possible set of sprints could have been:
\begin{enumerate}
  \item{Setting up all interfaces with dummy data to complete the pipeline}
  \item{Replace the dummy data with real data via implementing the necessary models or wrapping the available libararies}
  \item{Adding different levels to the game via image processing and implementing a more complex rigid body representation}
  \item{Allowing the user to draw the levels}
  \item{etc...}
\end{enumerate}
It can be observed that at the end of each sprint the entire code is running, integrated and producing some kind of output (despite its 	plausibility). During each sprint the output keeps developing and iteratively approaches its final shape. In addition the developers do not walk into the trap of being involved in their part while they are hoping that at some distant point in future, they will merge their code with the other teams' and everything will run smoothly. Moreover, this method allows for continuous unit-testing (explained in \autoref{unit_testing}) and early error correction.
One favourable aspect about agile management also is the resulting flexibility in human resources management through relocation and collocation. Relocation is the shuffling of one person's assignment from one team to another after completing the tasks in the respective team or if the team is no more in need of all the manpower. Collocation, on the other hand, is the distribution of one person among more than one team of developers, which is specially encouraged for people handling data interfaces between different teams.

\subsection{Useful Organizational Tools}
In this section, the light will be shed upon the methodology adopted by our group for deciding the project's theme and the concept of scrumming will be exposed. In addition, other facilitation tools will be presented.

\subsubsection{Project's Selection For Dummies}
Project selection is the first step in implementing it. With the flow of ideas and a wide spectrum of imagination among the developers on their first gatherings, they are suddenly confronted with a huge range of project ideas, that for the first glance seem appealing. However, project selection is more than just the vibrance of the idea. The following section will detail some tips on guiding the project selection.

\begin{itemize}
  \item \textbf{Brainstorming}: The first step in choosing a project topic is to set the sky as the limit for people to pour out their ideas all together into one pool, and most importantly, out-loud, allowing people to generate more ideas and develop one another's. An important aspect to be aware of is not to let this stage take a very long time, since time at Ferienakademie is somehow restricted. Also, it is be very helpful to find common grounds between different ideas as soon as possible. For example, "all the proposed ideas will require a Kinect input interface" or "regardless the topic we eventually select, we will need an fluid-structure-interaction framework". Reaching such conclusions early initiates the project in concurrency with finalizing/elaborating the ideas.
  
  \item \textbf{Selection Matrix}: It is very crucial and helpful to decide upon some criteria to evaluate the quality of the idea. On first sight one will assume that all ideas are creative and extravagant -- and it is of course desirable to have such an output -- but their charm might dim after being inspected subjectively with respect to criteria such as:
  \begin{itemize}
    \item \textit{Usability}: How easy can the end-user play this game or use this program? Will he/she need an intensive documentation or a direct supervised training or instructions to start using the game correctly?
    \item \textit{Realizability}: How good is the idea realizable? Can we reach a very simple first state with low effort?
    \item \textit{Extensibility}: How far can we go beyond the first milestone? Can we add more features to the project to make it more challenging and interesting?
    \item \textit{Audience-interest}: How attractive it is for audience to watch the game? For example, chess is a big fail even if you like it!
    \item \textit{Developers-interest}: Are we eager to put all our effort for around 10-days on that project?

  \end{itemize}
For our project, having a matrix (with projects' proposals on one edge and criteria on the other), the developers simply agreed a grade for each criterion for every idea and points we counted.
Automatically, some ideas dropped out and it was obvious towards which idea(s) the decision is heading. In case more than one project got close high ranks, their ideas might be re-discussed and re-evaluated for the final decision. If one cannot come up with a decision a hike often helps.
\end{itemize}


\subsubsection{Scrumming}
The term scrumming in agile management is inspired by the first move in an American football game, where all the players from each team initially start at the centerline and then in a glance spread all over the field - that move is the \emph{scrum}. The idealisation of such a concept in software development is such that the developers meet frequently (usually weekly or biweekly, at Ferienakademie of course more frequently) to discuss the following three points:
\begin{itemize}
  \item{What has been achieved/completed since the last scrum?}
  \item{What is the prospective task until the next scrum?}
  \item{Have any challenges popped up that require help from other developers?}
\end{itemize}
The main idea of scrumming is to keep all developers and the manger(s) aware of the current status of each separate area in the project and of what each team has achieved so far. 
Scrums are neither for feedback on the project or the developers nor for brainstorming. 
In our case, the scrum was held almost every morning, and if needed, it was rescheduled to a different time during the day, but generally we had a daily scrum.

\subsubsection{Code Handling Facilitators}
One of the duties of the project management is to ensure that code is being handled smoothly within the team. Our team has found the following tools pretty beneficial while working on the project.

\begin{itemize}
  \item \textbf{Git}\footnote{An alternative code management tool could be SVN}: for organized code sharing without conflicts, repository updating with current work, experimenting safely with developers' ideas on separate branches, managing conflicts between coders' work, etc. (for managing merge conflicts we suggest \href{meld}{http://meldmerge.org/})
  \item \textbf{CMake}: for a more neat and modular compiling and packaging of the whole project from one parent directory. CMake generates native makefiles and workspaces that can be robustly used in the compiler environment.
  \item \textbf{Valgrind}: for debugging the code and hunting malicious code lines that result in segmentation faults and other nasty bugs
  \item \textbf{Unit-testing}\footnote{Boost provides a good unit testing tool for C++}: testing each block of code or every interface separately with hardcoded example data to make sure it is bug free and interacting safely with given input and providing sensible output before handling it to the other teams up- and downstream the pipeline\label{unit_testing}
  \item \textbf{Code References}: since we were in an internet-deprived area, having an offline chest of language references and library documentation is worth a fortune. Useful references are official library documentations, c++ reference and an offline repository of stackoverflow.com (this also could be done for libraries which provide only online documentation such as boost libraries)
\end{itemize}
