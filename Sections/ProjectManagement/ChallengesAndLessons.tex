\section{Challenges \& Lessons Learned}
Challenges are not bugs, they are features, and one should be indeed aware of their existence and do his/her best to avoid them or at worst be ready to handle them if they can't be detoured. This section will wrap up a general reflection about the project in general, shedding the light on some mistakes that we fall into, challenges that we faces and lessons that we have found useful to be documented and communicated to reader.

\begin{itemize}
  \item \textbf{Time constraint}: We had many hikes, plus there were a couple of social gatherings and events such as the tournaments and the cultural evening. This boiled down two weeks of work dedicated to the project to much less time. This aspect needs to be taken into account when setting the milestones and the timeline for the project.
  \item \textbf{Defining the project's topic}: Due to the multi-disciplinary background of the participants as well as various interests, the brainstorming session turns out really vibrant between different ideas. Attention should be drawn to common grounds and the manager(s) should at some point occlude the ongoing flow of ideas to agree upon something to start coding with.
  \item \textbf{Defining the milestones (to be really agile)}: One of the challenges faced in our projects was phrasing the first milestone deliverables thorough-enough. The first milestone deliverable should not have been just about "Create a game with the simplest fluid-structure interaction and only taking input from Kinect". Such a statement comprised too much of details to work on such as the level of details of rigid bodies capabilities. In addition, it required a comprehensive data handling system between different pipeline stages, which implied that each team should necessarily implement fully-working functionality in order to test the pipeline integrity (imagine how long this cold have taken !!).
  A more efficient/agile approach would have been to set the first milestone deliverable to be "building and closing the pipeline with dummy data". This would have crossed out a milestone from the list and given all the teams an insight about into the interaction with the preceding and proceeding pipeline stages.
  \item \textbf{Testing the code}: With around 20 developers, each one implementing a chunk, the code is very vulnerable to get broken by some mathematical fault in a function or a segmentation-fault in one group's code. This could be avoided -- up to some extent -- by unit-testing. Unit-testing is hard-coding an example in order to test the implemented methods and ensure that they are giving the expected response. Although unit-testing requires a considerable additional workload, it, on the other hand, saves the programmers a lot of debugging time in case something goes wrong. This was extremely crucial, for instance, in the interface between rigid bodies and LBM due to the massive amount of communication between the two codes.
  \item \textbf{Selective Scrumming}: Our team consisted of about 20 developers and having all those in one meeting was thought to be hectic. Thus, for every team, a representative was chosen to attend the scrum and give updates on behalf of his team. Nevertheless, involving the entire team in the scrum meeting turned out to be not as a bad as it was prospected, since it would provide a more detailed insight about the progress of each team (versus the general statement about the progress given by the representative). Additionally this encourages a more direct communication channel between members from different teams dealing with common issues (such as interfaces).  In our project, it would have helped to have the developers of the LBM team listen directly in scrums from the RB team and vice versa, as well as involving the management and other teams such a discussion.
  \item \textbf{Managers Dedication}: The idea of a manager as someone holding meetings every now and then is not exactly accurate. The manager role is to actually "run around and manage". One of the traps we got into was underestimating the importance of the managers' full dedication to observing the progress, communicating with the developers (aside from the scrum meetings) and evaluating the project status and planning ahead upon that. As a consequence, the two managers took part in different teams and got involved in the details of their work and this led to an unbalance in the attention given to the project as a whole. One further recommendation to managers (and team leaders as well) is to incorporate a skeptical attitude towards the feedback provided; not that the managers should not trust others but to be more questioning about the stuff claimed to be complete, working or got fixed. In "Grand Theft Boat - Sartnal", questioning the validity of LBM implementation or nagging about unit-testing the classes could have spared the team later a lot of time debugging and fixing the code.
  \item \textbf{Waterfalling}: Waterfall management style differs from agile management style in that it treats the project as a one whole and the tangible output is only delivered at the final stage. This could be inefficient in projects like ours, since waiting till the end without a tangible output might get the team demotivated as time passes. Also, the project becomes more rigid to adding or modifying milestone deliverables for later stages. The risk of waterfalling is usually high when the first milestone is not clearly stated or not clarified precisely to the developers. The risk evolves because they end up indulging into coding and extending functionality of their specific blocks(ex: making an elaborate rigid bodies class structure), while delaying integration of their respective parts with the others on the pipeline.
  \item \textbf{Speaking German}: Although the course's official language is English, nevertheless, one can't hold a group of 90\% of Germans to speak their mother language. In such an environment, non-formal meetings between developers (meaning small talks during coffee breaks, discussion over lunch, casual chats, etc.) encourage ideas flow and facilitate problem solving among developers as well as helps management to be updated with deeper details of the project's status. It is rather recommended that the managers' language skills in later courses is good enough to engage and comprehend such situations easily and not just rely on the English scrum meeting for udpates.
  \item \textbf{Forgetting The Game}: At the end of the day, our course's theme was "Simulated Physics for Interactive \emph{Games}". Accordingly, a team dedicated to designing a game, with different levels, challenges and scenarios is what differentiates such a theme from applying the physics behind fluid-structure interaction in a wind turbine design. Although it is not mandatory to have assigned people from day one dedicated to such task, the management and the candidates for such role should be on stand-by to get involved in such task.
  \item \textbf{Presentation Topics}: Since this course was designed to deliver a project rather than conduct presentations, the presentations were rather short to give room for coding. Some of them were also a bit theoretical or not so necessary in the scope of the project (such as Navier-stokes presentations, for instance). Also, some concepts and tools that were more demanding were overlooked or not in the presentations plan, such as debugging, resolving merge conflicts\footnote{A presentation was held later for each of those topics when the necessity for them was noticed}, the pipeline concept and SDL\footnote{SDL was the initially intended tool in the code skeleton for the visualization. On the other hand, OpenGL, a more powerful yet not as flexible was used instead.}
\end{itemize}